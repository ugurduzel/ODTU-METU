\documentclass[12pt]{article}
\usepackage[utf8]{inputenc}
\usepackage{float}
\usepackage{amsmath}
\usepackage{mathtools}
\DeclarePairedDelimiter{\ceil}{\lceil}{\rceil}



\usepackage[hmargin=3cm,vmargin=6.0cm]{geometry}
%\topmargin=0cm
\topmargin=-2cm
\addtolength{\textheight}{6.5cm}
\addtolength{\textwidth}{2.0cm}
%\setlength{\leftmargin}{-5cm}
\setlength{\oddsidemargin}{0.0cm}
\setlength{\evensidemargin}{0.0cm}



\begin{document}

\section*{Student Information } 
%Write your full name and id number between the colon and newline
%Put one empty space character after colon and before newline
Full Name :  Ugur Duzel\\
Id Number : 2171569 \\

% Write your answers below the section tags
\section*{Answer 1}
\hspace{2cm}Lets call this inequality P(n) \\
1) Base Step: \\
\vspace{0.1cm}
\hspace{5cm}P(1) : $$(\sum\limits_{k=1}^1 k)^2 \geq \sum\limits_{k=1}^1 k^2$$ $$1 \geq 1 $$
2) Inductive Step: Assume that P(z) is true where, $1 \leq z \leq n$. Show that P(z+1) is also true.
\vspace{0.1cm}
\hspace{5cm}P(z) : $$(\sum\limits_{k=1}^z k)^2 \geq \sum\limits_{k=1}^z k^2$$ $$1 \geq 1 $$

P(z) : $$(\frac{z(z+1)}{2})^2 \geq  1+2^2+3^2+...+z^2$$
P(z+1) : $$(\frac{(z+1)(z+2)}{2})^2 \geq  1+2^2+3^2+...+z^2+(z+1)^2$$
Since we assumed the first inequality true, we can safely add $(z+1)^2$ to both sides.
$$ \frac{z^2(z+1)^2}{4}+(z+1)^2 \geq \sum\limits_{k=1}^{z+1} k^2$$
$$ (z+1)^2(\frac{z^2}{4}+1) \geq \sum\limits_{k=1}^{z+1} k^2$$
$$\frac{(z+1)^2(z^2+4)}{4}=\frac{(z+1)^2((z+2)^2-2z)}{4} \geq \sum\limits_{k=1}^{z+1} k^2$$
$$\frac{(z+1)^2(z^2+4)}{4} \geq \sum\limits_{k=1}^{z+1} k^2$$
This proves that P(z+1) is also true.
\section*{Answer 2}
\subsubsection*{1)}
\hspace{1.1cm}Since 21 is a number that cannot be paired to make the summation of the pair equal to 42, i.e. there is only one 21. We should consider this question like we have 40 numbers, which are $1,\ 2,\ ...,\ 20,\ 22,\ 23 \ ...,\ 41$. 

\hspace{0.5cm}Now assume that we don't have two people choosing numbers, we have only one person. The 21st selection(consider the 40 numbers without 21) will pair up to summation of 42. In that case, using the Pigeonhole Principle we have 21 pigeons (numbers) and 20 holes. This means that there will be at least two pigeons in a hole. In a scientific notation we can show the principle as $\ceil{\frac{21}{20}}=2$. 

\hspace{0.5cm}Now regarding the question if Alice acts cleverly and selects 21, Bob will have the first selection among the 40 numbers $1,\ 2,\ ...,\ 20,\ 22,\ 23 \ ...,\ 41$. This means Bob will be making the 21st selection among this 40 numbers, i.e 22nd when we count Alice's first selection of 21. In conclusion, if Alice plays her best strategy, Bob is forced to lose. This means Alice will win.

\subsubsection*{2)}
Since the order of picked up numbers does not matter, this is just very short to write down one by one.\\
\{0,\ 0,\ 5\}, \{0,\ 1,\ 4\}, \{0,\ 2,\ 3\}, \{1,\ 1,\ 3\}, \{1,\ 2,\ 2\}

\subsubsection*{3)}
We should consider this problem as the problems with combination with repetition and we should think of it like we have two separators in order to obtain three nonnegative integers. $$1\ \ 1\ \ |\ \ 1\ \ 1\ \ |\ \ 1\ \ or\ \ 1\ \ |\ \ 1\ \  1\ \ 1\ \ 1\ \ |\ \ $$ \\
For example in for the first one our integers are 2, 2 and 1 respectively, for the second one they are 1, 4 and 0 respectively. However in this case integers are positive so we can do a little trick. We think like we have distributed three of the ones to each of $x_1,\ x_2,\ x_3$. Then we think it the same way, only now we have 2 ones and 2 separators. This means we have 4 places to select to 2 separators. C(4,2) will give us the correct solution.
\section*{Answer 3}
\begin{equation} 
\begin{split}
\hspace{-1cm}
\sum\limits_{k=0}^{n}a_kx^k(1-x)^{3n-2k} 
& =a_0(1-x)^{3n}+a_1x(1-x)^{3n-2}+a_2x^2(1-x)^{3n-4}+...+a_nx^n(1-x)^n\\
& =(1-x)^n(a_0(1-x)^{2n}+a_1x(1-x)^{2n-2}+a_2x^2(1-x)^{2n-4}+...+a_nx^n(1-x)^0)\\
& =(1-x)^n(a_0(1-2x+x^2)^{n}+a_1x(1-2x+x^2)^{n-1}+...+a_nx^n(1-2x+x^2)^0)\\
& =(1-x)^n(\frac{a_0}{3^0}3^0(1-2x+x^2)^{n}+\frac{a_1}{3^1}3^1x^1(1-2x+x^2)^{n-1}+...+\frac{a_n}{3^n}3^nx^n(1-2x+x^2)^0)\\
& =(1-x)^n(1-2x+x^2+3x)^n \qquad (\frac{a_0}{3^k}=C(n,0)\ i.e. \frac{a_r}{3^k}=C(n,r)\ \ k:0 \rightarrow n)\\ 
& =(1-x)^n(1+x+x^2)^n \\
& =(1-x^3)^n 
\end{split}
\end{equation}
Therefore it should be $a_r=3^rC(n,r)$ so that we can make the right-side look like a binomial expression of the left side.This way it can be equal to the left side of the corresponding equality.
\section*{Answer 4}
\vspace{0.5cm}
$a_n=a_n^{(h)}+a_n^{(P)}$ \\
\vspace{0.1cm}
\hspace{1cm}
To find homogeneous solution solve $a_n-4a_{n-1}+a_{n-2}+6a_{n-3}=0$ by writing the corresponding equation $\alpha^3-4\alpha^2+\alpha+6=0$. Solving this equation we will have the following roots, $\alpha_1=2,\ \alpha_2=3,\ \alpha_3=-1$. Then we will have, \\
$$a_n^{(h)}  = c_1(2)^n+c_2(3)^n+c_3(-1)^n$$ \\
To find particular solution, guess that $a_n^{(P)}=an+b$ and plug it into $a_n=4a_{n-1}-a_{n-2}-6a_{n-3}n-2$ \\
$$(an+b)-4(a(n-1)+b)+(a(n-2)+b)+6(a(n-3)+b)=n-2$$\\
$$a=\frac{1}{4},\ b=\frac{1}{2}$$\qquad so this means that $a_n^{(P)}$ is $$a_n^{(P)}=\frac{1}{4}n+\frac{1}{2}$$ \\
$$a_n  = c_1(2)^n+c_2(3)^n+c_3(-1)^n+\frac{1}{4}n+\frac{1}{2}$$\\
$$a_0=c_1+c_2+c_3+\frac{1}{2}=\frac{7}{2}$$\\
$$a_1=2c_1+3c_2-c_3+\frac{3}{4}=\frac{19}{4}$$\\
$$a_2=4c_1+9c_2+c_3+1=13$$\\
$$ c_1=1.67,\ c_2=0.50 \ and\ c_3=0.83$$\\
Then the solution to this recurrence relation becomes,\\
$$a_n=(1.67)(2)^n+(0.50)(3)^n+(0.83)(-1)^n+(0.25)n+(0.50) $$\\




\end{document}

​

