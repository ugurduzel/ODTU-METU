\documentclass[12pt]{article}
\usepackage[utf8]{inputenc}
\usepackage{float}
\usepackage{amsmath}


\usepackage[hmargin=3cm,vmargin=6.0cm]{geometry}
%\topmargin=0cm
\topmargin=-2cm
\addtolength{\textheight}{6.5cm}
\addtolength{\textwidth}{2.0cm}
%\setlength{\leftmargin}{-5cm}
\setlength{\oddsidemargin}{0.0cm}
\setlength{\evensidemargin}{0.0cm}

%misc libraries goes here


\begin{document}

\section*{Student Information } 
%Write your full name and id number between the colon and newline
%Put one empty space character after colon and before newline
Full Name : Ugur Duzel \\
Id Number :  2171569 \\

% Write your answers below the section tags
\section*{Answer 1}
\subsubsection*{a)}
\vspace{0.5cm}
\hspace{1.5cm} Assume that $x \in (A \cap B) $, then it means $x \in A$, $x \in B$ \vspace{0.1cm}\\
$(x \in A \rightarrow x \in (A \cup \overline{B})) \quad \land \quad$ 
$(x \in B \rightarrow x \in (\overline{A} \cup B))$ \\
$(x \in (A \cup \overline{B})  \land x \in (\overline{A} \cup B)) \rightarrow x \in ((A \cup \overline{B}) \cap  (\overline{A} \cup B))$ \vspace{0.2cm}\\
This means that if I take a $x$ from the set $A \cap B$, this x is also going to be in $(A \cup \overline{B}) \cap  (\overline{A} \cup B)$. This fact proves that $A \cap B$ is a subset of $(A \cup \overline{B}) \cap  (\overline{A} \cup B)$. 
\subsubsection*{b)}
\vspace{0.5cm}
\hspace{1.5cm} Assume that $x \in (\overline{A} \cap \overline{B}) $, then it means $x \in \overline{A}$, $x \in \overline{B}$ \vspace{0.1cm}\\
$(x \in \overline{B} \rightarrow x \in (A \cup \overline{B})) \quad \land \quad$ 
$(x \in \overline{A} \rightarrow x \in (\overline{A} \cup B))$ \\
$(x \in (A \cup \overline{B})  \land x \in (\overline{A} \cup B)) \rightarrow x \in ((A \cup \overline{B}) \cap  (\overline{A} \cup B))$ \vspace{0.2cm}\\
This means that if I take a $x$ from the set $\overline{A} \cap \overline{B}$, this x is also going to be in $(A \cup \overline{B}) \cap  (\overline{A} \cup B)$. This fact proves that $\overline{A} \cap \overline{B}$ is a subset of $(A \cup \overline{B}) \cap  (\overline{A} \cup B)$. 

\vspace{1cm}
\section*{Answer 2} 
$f:X \rightarrow YxZ$,\  $A \subseteq Y$,\  $B \subseteq Y$,\  $C \subseteq Z$  \vspace{0.3cm}\\
\vspace{0.1cm}
$\hspace{1cm}i)\quad A\ x\ C \subseteq Y\ x\ Z \qquad since\ A \subseteq Y, C \subseteq Z$\\
\vspace{0.1cm}
$\hspace{2cm}f^{-1}(AxC) \subseteq X$ \vspace{0.1cm}\\
\vspace{0.1cm}
$\hspace{1cm}\ \ \quad B\ x\ C \subseteq Y\ x\ Z \qquad since\ B \subseteq Y, C \subseteq Z$\\
\vspace{0.1cm}
$\hspace{2cm}f^{-1}(BxC) \subseteq X$\\
\vspace{0.1cm}
$\hspace{2cm}f^{-1}(AxC)  \cap f^{-1}(BxC) \subseteq X$ \vspace{0.3cm}\\
\vspace{0.1cm}
$\hspace{0.8cm}ii)\quad (A \subseteq Y) \land (B \subseteq Y) \rightarrow (A \cap B) \subseteq Y$, $\hspace{0.2cm}(A \cap B)xC \subseteq YxZ \qquad since\ C \subseteq Z$ \vspace{0.1cm}\\
\vspace{0.1cm}
$\hspace{2cm}f^{-1}((A \cap B)xC) \subseteq X$ \vspace{0.3cm}\\
\vspace{0.1cm}
\hspace{0.65cm}$iii)$ Showing that each side is a subset of the other side is enough, to prove the given equality. \\
$Case\ 1:$ \\
\vspace{0.1cm}
$\hspace{2cm}f^{-1}((A \cap B)xC) \subseteq f^{-1}(AxC)\cap f^{-1}(BxC)$ \\
\vspace{0.1cm}
$\hspace{2cm}x \in f^{-1}((A \cap B)xC) \subseteq X$ 
\begin{equation} 
\begin{split}
Assume\ that\ f(x) = y \rightarrow \ 
& y \in (A \cap B)\ x\ C  \\
& y \in (\ (AxC)\cap (BxC)\ ) \\
& (y \in (AxC))\ \land \ (y \in (BxC)) \\
& (\ f^{-1}(y) \in f^{-1}(AxC)\ )\ \land \ (\ f^{-1}(y) \in f^{-1}(BxC)\ ) \\
& (\ x \in f^{-1}(AxC)\ )\ \land \ (\ x \in f^{-1}(BxC)\ ) \\
& x \in (\ f^{-1}(AxC)\ \cap \ f^{-1}(BxC)\ )
\end{split}
\end{equation}
$Case\ 2:$ \\
\vspace{0.1cm}
$\hspace{2cm}f^{-1}(AxC)\cap f^{-1}(BxC) \subseteq f^{-1}((A \cap B)xC)$ \\
\vspace{0.1cm}
$\hspace{2cm}x \in (\ f^{-1}(AxC)\cap f^{-1}(BxC)\ ) \subseteq X$ \\
\vspace{0.1cm}
$\hspace{2cm}(x \in f^{-1}(AxC))  \land (x \in  f^{-1}(BxC))$ 
\begin{equation} 
\begin{split}
Assume\ that\ f(x) = y \rightarrow \
& y \in A\ x\ C  \ \ \ \ \ \ \ \ \ \ \ \ \ \ \ \ \ \ \ \ \ \ \ \ \ \ \ \ \ \ \ \ \ \ \ \ \ \ \ \ \ \ \ \ \ \ \ \ \ \ \   \\
& y \in B\ x\ C  \\
& y \in (\ (AxC)\cap (BxC)\ ) \\
& y \in \ (A\cap B)\ x\ C \\
& f^{-1}(y) \in \ f^{-1}((A\cap B)xC) \\
& x \in \ f^{-1}((A\cap B)xC) \\
\end{split}
\end{equation}
\vspace{0.5cm}
\hspace{1cm}
\textit{These two cases prove the given equality $f^{-1}((A \cap B)xC) = f^{-1}(AxC)\cap f^{-1}(BxC)$.}

\section*{Answer 3}
\subsubsection*{a)}
\hspace{2cm}Since $x^2+5$ cannot be smaller than 0 this function's domain is $R$. However, for $x=2\ and\ x=-2$,\ $f(2)=f(-2)=ln(9)$ this function is not one-to-one. $f(x)=ln(x^2+5)$ has its lowest value at $x=0$ and it is $ln(5)$ so it means that this function is not onto. This function is neither one-to-one, nor it is not onto.
\subsubsection*{b)}
\hspace{2cm}Since this function is in the form of $e^{g(x)}$, it cannot be negative. So this function is not onto. On the other hand, power of $x$ is an odd number, for every distinct $x$, $x^7$ will be different, so as, $e^{x^7}$. This means that $f(x)=e^{e^{x^7}}$ is one-to-one. This function is one-to-one but it is not onto.

\section*{Answer 4}
\subsubsection*{a)}
\hspace{2cm} Let $A\ = \ \{ a_1,\ a_2,\ a_3, \ ...,\ a_i, \ ... \}$ and $B \ = \ \{ b_1,\ b_2,\ b_3, \ ...,\ b_i, \ ... \}$. \\
\begin{equation} 
\begin{split}
A\ x\ B \  & = \{ (a_1,\ b_1),\ (a_1,\ b_2),\ (a_1,\ b_3),\ ...,\ (a_1,\ b_i),\ ... \\
 & \qquad (a_2,\ b_1),\ (a_2,\ b_2),\ (a_2,\ b_3),\ ...,\ (a_2,\ b_i),\ ... \\
 & \qquad (a_3,\ b_1),\ (a_3,\ b_2),\ (a_3,\ b_3),\ ...,\ (a_3,\ b_i),\ ... \\
 & \qquad . \\ & \qquad . \\
  & \qquad (a_i,\ b_1),\ (a_i,\ b_2),\ (a_i,\ b_3),\ ...,\ (a_i,\ b_i),\ ... \\
   & \qquad . \\ & \qquad . \hspace{6.3cm}  \}
\end{split}
\end{equation}
Using the method from textbook called "Zigzag Method", we can find a one-to-one correspondence for $A\ x\ B$ as following, 
$$
A\ x\ B \ = \{(a_1,b_1),\ (a_1,b_2),\ (a_2,b_1),\ (a_3,b_1),\ (a_2,b_2),\ (a_1,b_3),\ (a_1,b_4),\ (a_2,b_3),\ (a_3,b_2),\ (a_4,b_1),\ ...   \}
$$
\subsubsection*{b)}
\hspace{0.5cm}\ Using the Part-C of this question. Assume that B is countable, then we can easily say that A is countable too since subset of a countable set is also countable (Part-C). However, this contradicts with the given premise that A is uncountable. Proof by contradiction dictates that the statement, B is countable, is wrong. So B is uncountable.
\subsubsection*{c)}
\hspace{0.5cm}\ Consider $B=\{b_1,\ b_2,\ b_3,\ b_4,\ ...\}$ since $A$ is a subset of $B$ we can easily say that \\ 
$B=A\cup \{b'_1,\ b'_2,\ b'_3,\ ...\}$ since left side of equality is countable and $\{b'_1,\ b'_2,\ b'_3,\ ...\}$ part is countable $A$ must be countable as well so that we can find a one-to-one correspondence.\\
\vspace{0.1cm}
\hspace{0.5cm}For $B=\{b'_1,\ a_1,\ b'_2,\ a_2,\ b'_3,\ a_3,\ ...\}$\ (where $A=\{a_1,\ a_2,\ a_3,\ ...\}$), we can write the correspondence as $b_1=b'_1$,\ $b_2=a_1$,\ $b_3=b'_2$,\ $b_4=a_2$,\ $b_5=b'_3$,\ $b_6=a_3$. So A is countable.

\section*{Answer 5} 

If $f_1 (x)$ is $O(f_2 (x))$, then we can say that 
$$ |f_1 (x)| \leq c . |f_2 (x)|$$
\subsubsection*{a)}
$\hspace{0.5cm}\ ln|f_1(x)| \leq ln|c . f_2(x)|$ \\
\vspace{0.1cm}
$\hspace{0.5cm}ln|f_1(x)| \leq ln|c|\ +\ ln|f_2(x)| \leq ln|c|\ .\ ln|f_2(x)|\qquad$     (for which $ \frac{ln|f_2(x)|}{ln|f_2(x)|-1} \leq ln|c|$) \\
\vspace{0.1cm}
$\hspace{0.5cm}ln|f_1(x)| \leq  c' \ .\ ln |f_2(x)|$ \\
\vspace{0.1cm}
\hspace{0.5cm}So it can be written as $ln|f_1 (x)|$ is $O(ln|f_2 (x)|)$
\subsubsection*{b)}
In order to disprove this, there has to be at least one counter example. \\
Consider $f_1(x)=x^3+kx$\  (where k $\in$ R) There for $f_2(x)$ will be $f_2(x)=x^3$. They both are increasing functions. \\
Just to make sure as a double check, $$\lim_{x \to\infty}\frac{f_1(x)}{f_2(x)}=\lim_{x \to\infty}\frac{x^3+kx}{x^3}=\lim_{x \to\infty}\frac{1+\frac{k}{x^2}}{x^3}=0$$ This being equal to zero means that $f_1(x)$ is dominated by $f_2(x)$ \\
Now consider this,
$$\lim_{x \to\infty}\frac{3^{f_1(x)}}{3^{f_2(x)}}=\lim_{x \to\infty}\frac{3^{x^3+kx}}{3^{x^3}}=\lim_{x \to\infty}\frac{3^{x^3}.3^{kx}}{3^{x^3}}=\lim_{x \to\infty}3^{kx}=\infty$$
This means that $3^{f_1(x)}$ is not dominated by $3^{f_2(x)}$ so $3^{f_1(x)}=O(3^{f_2(x)})$ is not correct.

\section*{Answer 6}
\subsubsection*{a)}
$\hspace{2cm}i) \qquad x\ =\ y$ 
\begin{equation} 
\begin{split}
(3^x-1)mod(3^x-1)\ & = \ 3^{(x\ mod\ x)}-1 \\
0 & = 1-1
\end{split}
\end{equation}
$\hspace{1.8cm}ii) \qquad x\ <\ y$ 
\begin{equation} 
\begin{split}
(3^x-1)mod(3^y-1)\ & = \ 3^x-1 \\
(3^y-1) &\ |\ (3^x-1)-(3^x-1) \\
(3^y-1) &\ |\ 0
\end{split}
\end{equation}
$\hspace{1.65cm}iii) \qquad y\ <\ x$ \\
\vspace{0.1cm}
$\hspace{1.8cm}\ \ \qquad x\ mod\ y \equiv \ m\ (m \in Z^+,\ m<y),\ yk=m-x\ (k \in Z^+)$ 
\begin{equation} 
\begin{split}
(3^y-1)\ & |\ 3^m-3^x \\
(3^y-1)\ & |\ 3^x(3^{m-x}-1) \\
(3^y-1)\ & |\ 3^x(3^{yk}-1) 
\end{split}
\end{equation}
\hspace{2.8cm}
This is true since $(3^y-1)\ |\ (3^{yk}-1)$ \quad (for every $k \in Z^+)$ \vspace{0.5cm}\\
In conclusion, for every possible $x$ and $y$ combination where $x,y\in Z^+$ given statement is true.





\subsubsection*{b)}
$
gcd(a,0)=0 \\
gcd(a,b)=gcd(b,a\ mod\ b) \\
gcd(123,277)=
gcd(277,123)=
gcd(123,31)=
gcd(31,30)=
gcd(30,1)=
gcd(1,0)=1$

\end{document}

​

